% !TeX encoding = ISO-8859-1
% !TeX spellcheck = es_ES

\noindent
El siguiente archivo es llamado \texttt{mesh.c}, en �l se ilustran las tareas y la funcionalidad mallada de la red:

\begin{lstlisting}[style=C]
#include "mesh.h"
#include <lwip/netdb.h>
#include <sys/param.h>
#include "CRC.h"
#include "UART1.h"
#include "driver/gpio.h"
#include "driver/timer.h"
#include "esp_event.h"
#include "esp_system.h"
#include "esp_websocket_client.h"
#include "lwip/err.h"
#include "lwip/sockets.h"
#include "lwip/sys.h"
#include "math.h"
#include "meters_data.h"
#include "nvs_rotate.h"
#include "ram-heap.h"
#include "task_verify.h"
#include "tcpip_adapter.h"

static uint8_t tx_buf[TX_SIZE] = {
	0,
};
static uint8_t rx_buf[RX_SIZE] = {
	0,
};
static bool is_mesh_connected = false;
static mesh_addr_t mesh_parent_addr;
static int mesh_layer = -1;

static const char *MESH_TAG = "mesh_main";
static const char *TCP_TAG = "TCP_SERVER";

uint8_t SLAVE_ID;
uint16_t fconv;
uint16_t port;
uint64_t energy_ini;
SemaphoreHandle_t smfPulso = NULL;
SemaphoreHandle_t smfNVS = NULL;
bool creador = true;

tipo_de_medidor tipo;
uint32_t baud_rate;

/********* Tareas del Root ************/
void esp_mesh_tx_to_ext(void *arg)
{
	/*
	* Recibe la trama proveniente del nodo solicitado en la red mesh y
	* Env�a la trama mediante el socket tcp creado en tcp_server_task
	*
	*/
	
	mesh_addr_t from;
	mesh_data_t data;
	data.data = rx_buf;
	data.size = sizeof(rx_buf);
	data.proto = MESH_PROTO_BIN;
	int flag = 0, sendControl;
	esp_err_t error;
	char trama[tamBUFFER];
	
	INT_VAL len;
	while (esp_mesh_is_root())
	{
		error = esp_mesh_recv(&from, &data, portMAX_DELAY, &flag, NULL, 0);
		if (error == ESP_OK)
		{
			len.Val = rx_buf[4] + rx_buf[5] + 6;
			for (int i = 0; i < len.Val; i++)
			{
				trama[i] = (char)rx_buf[i];
			}
			for (int i = 0; i < len.Val; i++)
			{
				printf("trama[%d] = %02x\r\n", i, trama[i]);
			}
			sendControl = send(men, trama, len.Val, 0);
			if (sendControl < 0 || sendControl != len.Val)
			{
				ESP_LOGE("Tx to Ext", "Error in send");
			}
		}
	}
	ESP_LOGE(MESH_TAG, "Se elimino tarea TX EXT");
	vTaskDelete(NULL);
}
static void tcp_server_task(void *pvParameters)
{
	/*
	*	Tarea que maneja el servidor TCP, por aqu� se realizan conexiones con
	*redes externas
	*/
	
	char rx_buffer[128];
	char addr_str[128];
	int addr_family;
	int ip_protocol;
	int sock = 0;
	struct timeval timeout, timeout_listen;
	timeout.tv_sec = 20;
	timeout.tv_usec = 0;
	timeout_listen.tv_sec = 60 * 5;
	timeout_listen.tv_usec = 0;
	
	#ifdef CONFIG_IPV4
	struct sockaddr_in dest_addr;
	dest_addr.sin_addr.s_addr = htonl(INADDR_ANY);
	dest_addr.sin_family = AF_INET;
	dest_addr.sin_port = htons(port);
	addr_family = AF_INET;
	ip_protocol = IPPROTO_IP;
	inet_ntoa_r(dest_addr.sin_addr, addr_str, sizeof(addr_str) - 1);
	
	#else // IPV6
	struct sockaddr_in6 dest_addr;
	bzero(&dest_addr.sin6_addr.un, sizeof(dest_addr.sin6_addr.un));
	dest_addr.sin6_family = AF_INET6;
	dest_addr.sin6_port = htons(port);
	addr_family = AF_INET6;
	ip_protocol = IPPROTO_IPV6;
	inet6_ntoa_r(dest_addr.sin6_addr, addr_str, sizeof(addr_str) - 1);
	#endif
	
	int listen_sock =
	socket(addr_family, SOCK_STREAM, ip_protocol); // Crea el socket
	if (listen_sock < 0)
	{
		ESP_LOGE(TCP_TAG, "Unable to create socket: errno %d", errno);
	}
	
	if (setsockopt(listen_sock, SOL_SOCKET, SO_RCVTIMEO, (char *)&timeout_listen,
	sizeof(timeout_listen)) == 0)
	{
		printf("Tiempo Listening %d s\r\n", (uint32_t)timeout_listen.tv_sec);
	}
	
	ESP_LOGI(TCP_TAG, "Socket created");
	
	int err = bind(listen_sock, (struct sockaddr *)&dest_addr,
	sizeof(dest_addr)); // Asigna ip al socket
	if (err != 0)
	{
		ESP_LOGE(TCP_TAG, "Socket unable to bind: errno %d", errno);
	}
	ESP_LOGI(TCP_TAG, "Socket bound, port %d", port);
	
	while (esp_mesh_is_root())
	{
		err = listen(listen_sock, 1); // A la espera de las posibles conexiones
		if (err != 0)
		{
			ESP_LOGE(TCP_TAG, "Error occurred during listen: errno %d", errno);
			break;
		}
		ESP_LOGI(TCP_TAG, "Socket listening");
		
		struct sockaddr_in6 source_addr; // Large enough for both IPv4 or IPv6
		uint addr_len = sizeof(source_addr);
		sock =
		accept(listen_sock, (struct sockaddr *)&source_addr,
		&addr_len); // Se acepta la conexi�n en caso de poder realizarse
		
		if (setsockopt(sock, SOL_SOCKET, SO_RCVTIMEO, (char *)&timeout,
		sizeof(timeout)) == 0)
		{
			printf("Configurado Timeout %d s\r\n", (uint32_t)timeout.tv_sec);
		}
		men = sock;
		
		if (sock < 0)
		{
			ESP_LOGE(TCP_TAG, "Unable to accept connection: errno %d", errno);
			continue;
		}
		ESP_LOGI(TCP_TAG, "Socket accepted");
		
		while (esp_mesh_is_root())
		{
			printf("Recibiendo TCP...\r\n");
			int len = recv(sock, rx_buffer, sizeof(rx_buffer) - 1,
			0); // Se queda esperando en el estado de recepci�n
			
			// Error occurred during receiving
			if (len < 0)
			{
				ESP_LOGE(TCP_TAG, "recv failed: errno %d", errno);
				timer_pause(TIMER_GROUP_0, TIMER_0);
				
				break;
			}
			
			// Connection closed
			else if (len == 0)
			{
				ESP_LOGI(TCP_TAG, "Connection closed");
				break;
			}
			// Data received
			else
			{
				// Get the sender's ip address as string
				if (source_addr.sin6_family == PF_INET)
				{
					inet_ntoa_r(((struct sockaddr_in *)&source_addr)->sin_addr.s_addr,
					addr_str, sizeof(addr_str) - 1);
				}
				else if (source_addr.sin6_family == PF_INET6)
				{
					inet6_ntoa_r(source_addr.sin6_addr, addr_str, sizeof(addr_str) - 1);
				}
				
				rx_buffer[len] =
				0; // Null-terminate whatever we received and treat like a string
				ESP_LOGI(TCP_TAG, "Received %d bytes from %s:", len, addr_str);
				ESP_LOGI(TCP_TAG, "%s", rx_buffer);
				
				xQueueSendToFront(RxSocket, rx_buffer, pdMS_TO_TICKS(1000));
			}
		}
		
		if (sock != -1)
		{
			ESP_LOGE(TCP_TAG, "Shutting down socket and restarting...");
			shutdown(sock, 0);
			close(sock);
			// socket(addr_family, SOCK_STREAM, ip_protocol);
		}
	}
	ESP_LOGE(MESH_TAG, "Se Elimino la tarea TCP\r\n");
	close(sock);
	vTaskDelete(NULL);
}
void esp_mesh_p2p_tx_main(void *Pa)
{
	/*
	* Tarea para repartir el mensaje recibido en el servidor por Broadcast de la
	* red Mesh
	*/
	
	esp_err_t err;
	mesh_data_t data;
	char mensaje2[tamBUFFER] = "";
	data.proto = MESH_PROTO_BIN;
	data.data = tx_buf;
	data.size = sizeof(tx_buf);
	data.tos = MESH_TOS_P2P;
	
	int tamano, len = 0;
	mesh_addr_t rt[CONFIG_MESH_ROUTE_TABLE_SIZE];
	
	while (esp_mesh_is_root())
	{
		xQueueReceive(RxSocket, (char *)mensaje2, portMAX_DELAY);
		
		tamano = 6 + (uint8_t)mensaje2[4] + (uint8_t)mensaje2[5];
		
		for (int i = 0; i < tamano; i++)
		{
			tx_buf[i] = (uint8_t)mensaje2[i];
		}
		esp_mesh_get_routing_table((mesh_addr_t *)&rt,
		CONFIG_MESH_ROUTE_TABLE_SIZE * 6, &len);
		
		printf("Mandando a Nodo\r\n");
		
		for (int i = 1; i < len; i++)
		{
			err = esp_mesh_send(&rt[i], &data, MESH_DATA_P2P, NULL, 0);
			if (err != ESP_OK)
			{
				ESP_LOGW(MESH_TAG, "\r\nMensaje no enviado\r\n");
			}
		}
	}
	ESP_LOGE(MESH_TAG, "Se elimino tarea Tx main\r\n");
	vTaskDelete(NULL);
}

/********* Tareas de un Nodo ************/
/**** Bus RS485 Standard ****/
void esp_mesh_p2p_rx_main(void *arg)
{
	INT_VAL len;
	INT_VAL CRC;
	esp_err_t err;
	mesh_addr_t from;
	mesh_data_t data;
	int flag = 0;
	data.data = rx_buf;
	data.size = sizeof(rx_buf);
	mesh_rx_pending_t pendientes, auxi = {.toSelf = 0};
	uint8_t trama[tamBUFFER];
	const unsigned char *aux = &trama[4];
	
	while (!esp_mesh_is_root())
	{
		printf("Esperando...\r\n");
		err = esp_mesh_recv(&from, &data, portMAX_DELAY, &flag, NULL, 0);
		esp_mesh_get_rx_pending(&pendientes);
		if (pendientes.toSelf != auxi.toSelf)
		{
			printf("Pendientes = %d\r\n", pendientes.toSelf);
			auxi.toSelf = pendientes.toSelf;
		}
		switch (err)
		{
			case ESP_OK:
			{
				printf("Recibido por Mesh\r\n");
				
				if (!esp_mesh_is_root())
				{
					len.Val = rx_buf[4] + rx_buf[5];
					
					for (uint16_t i = 0; i < len.Val; i++)
					{
						trama[i + 4] = rx_buf[6 + i];
					}
					CRC.Val = CRC16(aux, len.Val);
					len.Val = len.Val + 2;
					trama[len.Val + 2] = CRC.byte.LB;
					trama[len.Val + 3] = CRC.byte.HB;
					trama[0] = rx_buf[0];
					trama[1] = rx_buf[1];
					trama[2] = len.byte.HB;
					trama[3] = len.byte.LB;
					xQueueSendToFront(TxRS485, &trama, portMAX_DELAY);
				}
			}
			break;
			case ESP_ERR_MESH_NOT_START:
			{
				ESP_LOGE(MESH_TAG, "Aun no esta Iniciada la Mesh\r\n");
			}
			break;
			case ESP_ERR_MESH_TIMEOUT:
			{
				ESP_LOGW(MESH_TAG, "Timeout Error\r\n");
			}
			break;
			default:
			ESP_ERROR_CHECK_WITHOUT_ABORT(err);
			break;
		}
	}
	
	ESP_LOGW(MESH_TAG, "Tarea Rx main eliminada");
	vTaskDelete(NULL);
}
void bus_rs485(void *arg)
{
	INT_VAL longitud;
	mesh_data_t dataMesh;
	BaseType_t ctrl_cola;
	
	dataMesh.proto = MESH_PROTO_BIN;
	dataMesh.size = tamBUFFER;
	dataMesh.data = tx_buf;
	dataMesh.tos = MESH_TOS_P2P;
	
	uint16_t txlen, txctrl;
	
	uint8_t dataTx[tamBUFFER];
	uint8_t *aux = &dataTx[4];
	
	while (!esp_mesh_is_root())
	{
		// Espera a recibir la trama RTU
		
		xQueueReceive(TxRS485, (uint8_t *)dataTx, portMAX_DELAY);
		printf("Recibido en cola...\r\n");
		txlen = dataTx[2] + dataTx[3];
		
		txctrl = uart_write_bytes(uart1, (char *)aux, txlen);
		
		if (txctrl > 0)
		{
			ESP_LOGI("RS485", "Tx FIFO:%u", txctrl);
			ctrl_cola =
			xQueueReceive(RxRS485, (uint8_t *)tx_buf + 4, pdMS_TO_TICKS(10000));
			if (ctrl_cola == pdTRUE)
			{
				tx_buf[0] = dataTx[0];
				tx_buf[1] = dataTx[1];
				tx_buf[2] = 0;
				tx_buf[3] = 0;
				longitud.Val = tx_buf[4] + tx_buf[5] - 2;
				tx_buf[4] = longitud.byte.HB;
				tx_buf[5] = longitud.byte.LB;
				tx_buf[4 + longitud.Val] = 0x00;
				tx_buf[4 + longitud.Val + 1] = 0x00;
				for (int i = 0; i < (longitud.Val + 4); i++)
				{
					printf("tx_buf[%d] = %02x\r\n", i, tx_buf[i]);
				}
				esp_err_t err = esp_mesh_send(NULL, &dataMesh, MESH_DATA_P2P, NULL, 0);
				if (err == ESP_OK)
				{
					printf("Mandado\r\n");
				}
			}
			else
			{
				ESP_LOGW("RS485", "Queue timeout proceeding to UART1 Flush");
				uart_flush(UART_NUM_1);
			}
		}
	}
	vTaskDelete(NULL);
}

/**** Medidor de pulsos *****/
/*Interrupcion de entrada de pulso*/
void IRAM_ATTR INT_GPIO_PULSOS(void *arg)
{
	xSemaphoreGiveFromISR(smfPulso, NULL);
}
void modbus_tcpip_pulsos(void *arg)
{
	esp_err_t err;
	int flag = 0;
	mesh_addr_t from;
	mesh_data_t data_rx, data_tx;
	data_rx.data = rx_buf;
	data_rx.size = RX_SIZE;
	data_tx.data = tx_buf;
	data_tx.size = TX_SIZE;
	
	energytype_t energia;
	
	while (!esp_mesh_is_root())
	{
		ESP_LOGI("MODBUS", "Entr� aqu�");
		err = esp_mesh_recv(&from, &data_rx, portMAX_DELAY, &flag, NULL, 0);
		ESP_LOGI("MODBUS", "Mensaje Recibido");
		ESP_LOGI("MODBUS RX", "%x %x %x %x %x %x %x %x %x %x %x %x", rx_buf[0],
		rx_buf[1], rx_buf[2], rx_buf[3], rx_buf[4], rx_buf[5], rx_buf[6],
		rx_buf[7], rx_buf[8], rx_buf[9], rx_buf[10], rx_buf[11]);
		
		if ((err == ESP_OK) && (rx_buf[6] == SLAVE_ID) && (rx_buf[7] == 0x03) &&
		(rx_buf[8] == MODBUS_ENERGY_REG_INIT_POS_H) &&
		(rx_buf[9] == MODBUS_ENERGY_REG_INIT_POS_L) &&
		(rx_buf[11] == MODBUS_ENERGY_REG_LEN))
		{
			xQueuePeek(Cuenta_de_pulsos, &(energia.tot), portMAX_DELAY);
			tx_buf[0] = rx_buf[0];
			tx_buf[1] = rx_buf[1];
			tx_buf[2] = 0x00;
			tx_buf[3] = 0x00;
			tx_buf[4] = 0x00;     // len hb
			tx_buf[5] = 0x0b;     // len lb
			tx_buf[6] = SLAVE_ID; // SlaveID
			tx_buf[7] = 0x03;
			tx_buf[8] = 0x08; // byte count
			tx_buf[9] = energia.u8.lll8;
			tx_buf[10] = energia.u8.llll8;
			tx_buf[11] = energia.u8.l8;
			tx_buf[12] = energia.u8.ll8;
			tx_buf[13] = energia.u8.hh8;
			tx_buf[14] = energia.u8.h8;
			tx_buf[15] = energia.u8.hhhh8;
			tx_buf[16] = energia.u8.hhh8;
			err = esp_mesh_send(NULL, &data_tx, MESH_DATA_P2P, NULL, 0);
			if (err == ESP_OK)
			{
				ESP_LOGI("MODBUS", "Respuesta enviada");
			}
		}
		ESP_LOGI("MODBUS", "Final del ciclo");
	}
	vTaskDelete(NULL);
}
static void rotar_nvs(void *arg)
{
	/* Nomenclatura:
	* 	pf: P�gina fija. Es la pagina que necesita acceder el micro para saber
	*en qu� p�gina variable se encuentra. en ella se encuentra un registro de
	*igual nombre (pf) que contiene el n�mero de paginas variables llenas.
	*
	* 	pv: P�gina variable. En esta p�gina se guarda la cuenta de cada pulso
	*registrado. Las p�ginas variables como su nombre lo indica van variando a
	*medida que crece la cuenta de pulsos.
	*
	*	partition: Espacio de memoria flash del micro en la que se encuentra
	*actualmente el �ltimo pulso.
	*/
	
	uint8_t partition_number; // Index de partition name (app#)
	
	int32_t indexPv,        // Index del namespace (pv#)
	indexEntradaActual, // Index de la entrada activa en la pv (e#)
	cuentaPulsosPv;     // Cuenta de pulsos en la entrada activa
	
	uint64_t total_pulsos, // Total de pulsos: suma de lops pulsos en cada
	// particion, p�gina y entrada utilizada
	inicial_pulsos;
	char *pvActual,       // String para namespace de la p�gina variable
	*entradaActualpv, // String para el key de la entrada actual
	*partition_name;  // String para el nombre de la partici�n
	
	inicial_pulsos = round((float)energy_ini * (float)fconv);
	xQueueOverwrite(Cuenta_de_pulsos, &total_pulsos);
	
	esp_err_t err = search_init_partition(&partition_number);
	
	if (asprintf(&partition_name, "app%d", partition_number) < 0)
	{
		free(partition_name);
		ESP_LOGE("ROTAR_NVS", "Nombre de particion no fue creado");
	}
	
	nvs_flash_init_partition(partition_name);
	
	show_ram_status("Partici�n iniciada");
	
	/*
	* Leyendo el valor del contador de la pagina fija. Retorna el namespace y el
	* n�mero de pagina variable se encuentran los datos m�s recientes
	*/
	err = leer_contador_pf(&partition_name, &pvActual, &indexPv);
	if (err != ESP_OK)
	printf("Error (%s) leyendo contador desde la pagina fija!\n",
	esp_err_to_name(err));
	ESP_LOGI("LPF", "pvActual: %s indexPv: %d", pvActual, indexPv);
	
	ESP_LOGI("DEBUG", "Antes de leer pagina variable");
	
	/*
	* Revisar pagina variable actual y buscar el ultimo registro escrito
	*/
	err = leer_pagina_variable(&partition_name, &pvActual, &indexEntradaActual,
	&entradaActualpv, &cuentaPulsosPv);
	if (err != ESP_OK)
	printf("Error (%s) buscando el registro escrito m�s reciente\n",
	esp_err_to_name(err));
	ESP_LOGI("LPV",
	"pvActual: %s indexEntradaActual: %d Entrada actual: %s Cuenta "
	"pulsos: %d",
	pvActual, indexEntradaActual, entradaActualpv, cuentaPulsosPv);
	
	ESP_LOGI("DEBUG", "Antes del while");
	
	while (!esp_mesh_is_root())
	{
		// Interrupci�n generada por pulsos
		if (xSemaphoreTake(smfPulso, portMAX_DELAY) == pdTRUE)
		{
			/*Si se lleg� al l�mite en una particion*/
			if (indexPv == Limite_paginas_por_particion &&
			indexEntradaActual == Limite_entradas_por_pagina &&
			cuentaPulsosPv == Limite_pulsos_por_entrada)
			{
				partition_number++;
				
				if (partition_number <= max_particiones)
				{
					levantar_bandera(partition_name);
					
					char *aux;
					nvs_handle_t my_handle;
					
					// Cerrando la particion anterior
					if (asprintf(&aux, "app%d", partition_number - 1) < 0)
					{
						free(aux);
						ESP_LOGE("ROTAR_NVS", "Nombre de particion no fue creado");
					}
					ESP_LOGI("DEINIT", "%s", aux);
					err = nvs_flash_deinit_partition(aux);
					if (err != ESP_OK)
					ESP_LOGE("CNP", "ERROR (%s) IN DEINIT", esp_err_to_name(err));
					else
					free(aux);
					
					// Inicializando la nueva partici�n
					if (asprintf(&partition_name, "app%u", partition_number) < 0)
					{
						free(partition_name);
						ESP_LOGE("ROTAR_NVS", "Nombre de particion no fue creado");
					}
					ESP_LOGW("CNP", "Partition changed to %s", partition_name);
					nvs_flash_init_partition(partition_name);
					
					ESP_LOGI("PARTICION ACTUAL", "DEBUG 2 %s", partition_name);
					
					// Llenando particion
					esp_err_t err = nvs_open_from_partition(partition_name, "storage",
					NVS_READWRITE, &my_handle);
					if (err != ESP_OK)
					ESP_LOGE("NVS", "ERROR IN NVS_OPEN");
					//		free(*pname);
					
					// Colocando la bandera de llenado en 0
					err = nvs_set_u8(my_handle, "finished", 0);
					if (err != ESP_OK)
					ESP_LOGE("NVS", "ERROR IN SET");
					err = nvs_commit(my_handle);
					if (err != ESP_OK)
					ESP_LOGE("NVS", "ERROR IN COMMIT");
					
					// Get del n�mero de la partici�n
					err = nvs_get_u8(my_handle, "pnumber", &partition_number);
					if (err != ESP_OK && err != ESP_ERR_NVS_NOT_FOUND)
					ESP_LOGE("NVS", "ERROR IN GET");
					else if (err == ESP_ERR_NVS_NOT_FOUND)
					{
						err = nvs_set_u8(my_handle, "pnumber", partition_number);
						if (err != ESP_OK)
						ESP_LOGE("NVS", "ERROR IN SET");
						err = nvs_commit(my_handle);
						if (err != ESP_OK)
						ESP_LOGE("NVS", "ERROR IN COMMIT");
					}
					// Close
					nvs_close(my_handle);
				}
			}
			
			if (partition_number <= max_particiones)
			{
				err = contar_pulsos_nvs(&partition_name, &partition_number, &indexPv,
				&indexEntradaActual, &cuentaPulsosPv);
				
				total_pulsos =
				inicial_pulsos + // Aporte de los pulsos iniciales
				(partition_number - 1) * Limite_paginas_por_particion *
				Limite_entradas_por_pagina *
				Limite_pulsos_por_entrada + // Aporte c/u de las particiones
				(indexPv - 1) * Limite_entradas_por_pagina *
				Limite_pulsos_por_entrada + // Aporte de las paginas anteriores
				(indexEntradaActual - 1) *
				Limite_pulsos_por_entrada + // Aporte de la p�gina actual
				cuentaPulsosPv;                 // Aporte de la entrada actual
				
				ESP_LOGI("PULSOS", "PARTICION: %d PV: %d ENT: %d Pulsos: %d",
				partition_number, indexPv, indexEntradaActual, cuentaPulsosPv);
				ESP_LOGI("TOTAL PULSOS", "%llu", total_pulsos);
				
				// Enviando los pulsos a otra tarea
				xQueueOverwrite(Cuenta_de_pulsos, &total_pulsos);
			}
			else
			ESP_LOGE("APP", "All partitions are full");
			show_ram_status("Por pulsos");
		}
	}
	
	free(pvActual);
	free(entradaActualpv);
	free(partition_name);
	ESP_LOGE("RTOS", "La tarea NVS-Rotative ha sido eliminada");
	vTaskDelete(NULL);
}

/**** Manejadores de eventos ****/
void mesh_event_handler(void *arg, esp_event_base_t event_base,
int32_t event_id, void *event_data)
{
	char rx_child[9] = "Rx_child";
	char rx_rs485[9] = "Rx_RS485";
	char modbus_pulse[11] = "Commun P2P";
	char count_pulse[] = "Rotative NVS";
	mesh_addr_t id = {
		0,
	};
	static uint8_t last_layer = 0;
	
	switch (event_id)
	{
		case MESH_EVENT_STARTED:
		{
			esp_mesh_get_id(&id);
			ESP_LOGI(MESH_TAG, "<MESH_EVENT_MESH_STARTED>ID:" MACSTR "",
			MAC2STR(id.addr));
			is_mesh_connected = false;
			mesh_layer = esp_mesh_get_layer();
		}
		break;
		case MESH_EVENT_STOPPED:
		{
			ESP_LOGI(MESH_TAG, "<MESH_EVENT_STOPPED>");
			is_mesh_connected = false;
			mesh_layer = esp_mesh_get_layer();
		}
		break;
		case MESH_EVENT_CHILD_CONNECTED:
		{
			mesh_event_child_connected_t *child_connected =
			(mesh_event_child_connected_t *)event_data;
			ESP_LOGI(MESH_TAG, "<MESH_EVENT_CHILD_CONNECTED>aid:%d, " MACSTR "",
			child_connected->aid, MAC2STR(child_connected->mac));
		}
		break;
		case MESH_EVENT_CHILD_DISCONNECTED:
		{
			mesh_event_child_disconnected_t *child_disconnected =
			(mesh_event_child_disconnected_t *)event_data;
			ESP_LOGI(MESH_TAG, "<MESH_EVENT_CHILD_DISCONNECTED>aid:%d, " MACSTR "",
			child_disconnected->aid, MAC2STR(child_disconnected->mac));
		}
		break;
		case MESH_EVENT_ROUTING_TABLE_ADD:
		{
			mesh_event_routing_table_change_t *routing_table =
			(mesh_event_routing_table_change_t *)event_data;
			ESP_LOGW(MESH_TAG, "<MESH_EVENT_ROUTING_TABLE_ADD>add %d, new:%d",
			routing_table->rt_size_change, routing_table->rt_size_new);
		}
		break;
		case MESH_EVENT_ROUTING_TABLE_REMOVE:
		{
			mesh_event_routing_table_change_t *routing_table =
			(mesh_event_routing_table_change_t *)event_data;
			ESP_LOGW(MESH_TAG, "<MESH_EVENT_ROUTING_TABLE_REMOVE>remove %d, new:%d",
			routing_table->rt_size_change, routing_table->rt_size_new);
		}
		break;
		case MESH_EVENT_NO_PARENT_FOUND:
		{
			mesh_event_no_parent_found_t *no_parent =
			(mesh_event_no_parent_found_t *)event_data;
			ESP_LOGI(MESH_TAG, "<MESH_EVENT_NO_PARENT_FOUND>scan times:%d",
			no_parent->scan_times);
		}
		/* TODO handler for the failure */
		break;
		case MESH_EVENT_PARENT_CONNECTED:
		{
			gpio_set_level(LED_PAPA, 1);
			mesh_event_connected_t *connected = (mesh_event_connected_t *)event_data;
			esp_mesh_get_id(&id);
			mesh_layer = connected->self_layer;
			memcpy(&mesh_parent_addr.addr, connected->connected.bssid, 6);
			ESP_LOGI(
			MESH_TAG,
			"<MESH_EVENT_PARENT_CONNECTED>layer:%d-->%d, parent:" MACSTR
			"%s, ID:" MACSTR "",
			last_layer, mesh_layer, MAC2STR(mesh_parent_addr.addr),
			esp_mesh_is_root() ? "<ROOT>" : (mesh_layer == 2) ? "<layer2>"
			: "",
			MAC2STR(id.addr));
			last_layer = mesh_layer;
			is_mesh_connected = true;
			if (esp_mesh_is_root())
			{
				tcpip_adapter_dhcpc_start(TCPIP_ADAPTER_IF_STA);
			}
			else
			{
				switch (tipo)
				{
					case (rs485):
					creador = vTaskB(rx_child);
					if (creador)
					{
						xTaskCreatePinnedToCore(esp_mesh_p2p_rx_main, rx_child, 3072 * 2,
						NULL, 5, NULL, 0);
					}
					creador = vTaskB(rx_rs485);
					if (creador)
					{
						xTaskCreatePinnedToCore(bus_rs485, rx_rs485, 3072 * 3, NULL, 5,
						NULL, 1);
						iniciarUART(tipo, baud_rate);
					}
					break;
					case (pulsos):
					creador = vTaskB(modbus_pulse);
					if (creador)
					{
						xTaskCreatePinnedToCore(modbus_tcpip_pulsos, modbus_pulse,
						3072 * 2, NULL, 5, NULL, 0);
					}
					creador = vTaskB(count_pulse);
					if (creador)
					xTaskCreatePinnedToCore(rotar_nvs, count_pulse, 4 * 1024, NULL, 5,
					NULL, 1);
					break;
					case (chino):
					creador = vTaskB(rx_child);
					if (creador)
					{
						xTaskCreatePinnedToCore(esp_mesh_p2p_rx_main, rx_child, 3072 * 2,
						NULL, 5, NULL, 0);
					}
					creador = vTaskB(rx_rs485);
					if (creador)
					{
						xTaskCreatePinnedToCore(bus_rs485, rx_rs485, 3072 * 3, NULL, 5,
						NULL, 1);
						iniciarUART(tipo, baud_rate);
					}
					break;
					case (enlace):
					break;
				}
			}
		}
		break;
		case MESH_EVENT_PARENT_DISCONNECTED:
		{
			gpio_set_level(LED_PAPA, 0);
			mesh_event_disconnected_t *disconnected =
			(mesh_event_disconnected_t *)event_data;
			ESP_LOGI(MESH_TAG, "<MESH_EVENT_PARENT_DISCONNECTED>reason:%d",
			disconnected->reason);
			is_mesh_connected = false;
			mesh_layer = esp_mesh_get_layer();
		}
		break;
		case MESH_EVENT_LAYER_CHANGE:
		{
			mesh_event_layer_change_t *layer_change =
			(mesh_event_layer_change_t *)event_data;
			mesh_lay
		}er = layer_change->new_layer;
			ESP_LOGI(
			MESH_TAG, "<MESH_EVENT_LAYER_CHANGE>layer:%d-->%d%s", last_layer,
			mesh_layer,
			esp_mesh_is_root() ? "<ROOT>" : (mesh_layer == 2) ? "<layer2>"
			: "");
			last_layer = mesh_layer;
		}
		break;
		case MESH_EVENT_ROOT_ADDRESS:
		{
			mesh_event_root_address_t *root_addr =
			(mesh_event_root_address_t *)event_data;
			ESP_LOGI(MESH_TAG, "<MESH_EVENT_ROOT_ADDRESS>root address:" MACSTR "",
			MAC2STR(root_addr->addr));
			for (int i = 0; i < 6; i++)
			{
				root_address.addr[i] = root_addr->addr[i];
			}
		break;
		case MESH_EVENT_VOTE_STARTED:
		{
			mesh_event_vote_started_t *vote_started =
			(mesh_event_vote_started_t *)event_data;
			ESP_LOGI(
			MESH_TAG,
			"<MESH_EVENT_VOTE_STARTED>attempts:%d, reason:%d, rc_addr:" MACSTR "",
			vote_started->attempts, vote_started->reason,
			MAC2STR(vote_started->rc_addr.addr));
		}
		break;
		case MESH_EVENT_VOTE_STOPPED:
		{
			ESP_LOGI(MESH_TAG, "<MESH_EVENT_VOTE_STOPPED>");
			break;
		}
		case MESH_EVENT_ROOT_SWITCH_REQ:
		{
			mesh_event_root_switch_req_t *switch_req =
			(mesh_event_root_switch_req_t *)event_data;
			ESP_LOGI(MESH_TAG,
			"<MESH_EVENT_ROOT_SWITCH_REQ>reason:%d, rc_addr:" MACSTR "",
			switch_req->reason, MAC2STR(switch_req->rc_addr.addr));
		}
		break;
		case MESH_EVENT_ROOT_SWITCH_ACK:
		{
			/* new root */
			mesh_layer = esp_mesh_get_layer();
			esp_mesh_get_parent_bssid(&mesh_parent_addr);
			ESP_LOGI(MESH_TAG,
			"<MESH_EVENT_ROOT_SWITCH_ACK>layer:%d, parent:" MACSTR "",
			mesh_layer, MAC2STR(mesh_parent_addr.addr));
		}
		break;
		case MESH_EVENT_TODS_STATE:
		{
			mesh_event_toDS_state_t *toDs_state =
			(mesh_event_toDS_state_t *)event_data;
			ESP_LOGI(MESH_TAG, "<MESH_EVENT_TODS_REACHABLE>state:%d", *toDs_state);
		}
		break;
		case MESH_EVENT_ROOT_FIXED:
		{
			mesh_event_root_fixed_t *root_fixed =
			(mesh_event_root_fixed_t *)event_data;
			ESP_LOGI(MESH_TAG, "<MESH_EVENT_ROOT_FIXED>%s",
			root_fixed->is_fixed ? "fixed" : "not fixed");
		}
		break;
		case MESH_EVENT_ROOT_ASKED_YIELD:
		{
			mesh_event_root_conflict_t *root_conflict =
			(mesh_event_root_conflict_t *)event_data;
			ESP_LOGI(MESH_TAG,
			"<MESH_EVENT_ROOT_ASKED_YIELD>" MACSTR ", rssi:%d, capacity:%d",
			MAC2STR(root_conflict->addr), root_conflict->rssi,
			root_conflict->capacity);
		}
		break;
		case MESH_EVENT_CHANNEL_SWITCH:
		{
			mesh_event_channel_switch_t *channel_switch =
			(mesh_event_channel_switch_t *)event_data;
			ESP_LOGI(MESH_TAG, "<MESH_EVENT_CHANNEL_SWITCH>new channel:%d",
			channel_switch->channel);
		}
		break;
		case MESH_EVENT_SCAN_DONE:
		{
			mesh_event_scan_done_t *scan_done = (mesh_event_scan_done_t *)event_data;
			ESP_LOGI(MESH_TAG, "<MESH_EVENT_SCAN_DONE>number:%d", scan_done->number);
		}
		break;
		case MESH_EVENT_NETWORK_STATE:
		{
			mesh_event_network_state_t *network_state =
			(mesh_event_network_state_t *)event_data;
			ESP_LOGI(MESH_TAG, "<MESH_EVENT_NETWORK_STATE>is_rootless:%d",
			network_state->is_rootless);
		}
		break;
		case MESH_EVENT_STOP_RECONNECTION:
		{
			ESP_LOGI(MESH_TAG, "<MESH_EVENT_STOP_RECONNECTION>");
		}
		break;
		case MESH_EVENT_FIND_NETWORK:
		{
			mesh_event_find_network_t *find_network =
			(mesh_event_find_network_t *)event_data;
			ESP_LOGI(MESH_TAG,
			"<MESH_EVENT_FIND_NETWORK>new channel:%d, router BSSID:" MACSTR
			"",
			find_network->channel, MAC2STR(find_network->router_bssid));
		}
		break;
		case MESH_EVENT_ROUTER_SWITCH:
		{
			mesh_event_router_switch_t *router_switch =
			(mesh_event_router_switch_t *)event_data;
			ESP_LOGI(MESH_TAG,
			"<MESH_EVENT_ROUTER_SWITCH>new router:%s, channel:%d, " MACSTR
			"",
			router_switch->ssid, router_switch->channel,
			MAC2STR(router_switch->bssid));
		}
		break;
		default:
		ESP_LOGI(MESH_TAG, "unknown id:%d", event_id);
		break;
	}
}
void ip_event_handler(void *arg, esp_event_base_t event_base, int32_t event_id,
void *event_data)
{
	char tcp_server[11] = "tcp_server";
	char tx_root[3] = "TX";
	char tx_ext[7] = "TX_EXT";
	ip_event_got_ip_t *event = (ip_event_got_ip_t *)event_data;
	ESP_LOGI(MESH_TAG, "<IP_EVENT_STA_GOT_IP>IP:%s",
	ip4addr_ntoa(&event->ip_info.ip));
	creador = vTaskB(tcp_server);
	if (creador)
	{
		xTaskCreatePinnedToCore(tcp_server_task, "tcp_server", 1024 * 4, NULL, 1,
		NULL, 1);
	}
	creador = vTaskB(tx_root);
	if (creador)
	{
		xTaskCreatePinnedToCore(esp_mesh_p2p_tx_main, "TX", 1024 * 3, NULL, 5, NULL,
		0);
	}
	creador = vTaskB(tx_ext);
	if (creador)
	{
		xTaskCreatePinnedToCore(esp_mesh_tx_to_ext, "TX_EXT", 3072 * 2, NULL, 5,
		NULL, 0);
	}
}

/****************************************/
/**** Configuracion e inicializacion ****/
/****************************************/
void config_gpio_pulsos(tipo_de_medidor tipo)
{
	if (tipo == pulsos)
	{
		ESP_LOGI(MESH_TAG, "Configurando GPIO para medidor tipo pulsos");
		// Tipo de interrupcion
		gpio_install_isr_service(ESP_INTR_FLAG_DEFAULT);
		
		gpio_pad_select_gpio(PULSOS);
		gpio_set_direction(PULSOS, GPIO_MODE_DEF_INPUT);
		gpio_isr_handler_add(PULSOS, INT_GPIO_PULSOS, NULL);
		gpio_set_intr_type(PULSOS, GPIO_INTR_POSEDGE);
	}
	else
	{
		ESP_LOGI(MESH_TAG, "Configurando GPIO para medidor tipo RS485");
		gpio_pad_select_gpio(RS485);
		gpio_set_direction(RS485, GPIO_MODE_DEF_OUTPUT);
	}
	gpio_pad_select_gpio(LED_PAPA);
	gpio_set_direction(LED_PAPA, GPIO_MODE_DEF_OUTPUT);
}
/*Inicio Mesh*/
void mesh_init(form_mesh fmesh, form_locwifi fwifi, form_modbus fmodbus)
\end{lstlisting}

\noindent
Ahora compila usando \texttt{gcc}:

\begin{listing}[style=consola, numbers=none]
	$ gcc  -o hello hello.c
\end{listing}